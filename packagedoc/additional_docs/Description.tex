% Copyright 2020-2023 Robert Bosch GmbH

% Licensed under the Apache License, Version 2.0 (the "License");
% you may not use this file except in compliance with the License.
% You may obtain a copy of the License at

% http://www.apache.org/licenses/LICENSE-2.0

% Unless required by applicable law or agreed to in writing, software
% distributed under the License is distributed on an "AS IS" BASIS,
% WITHOUT WARRANTIES OR CONDITIONS OF ANY KIND, either express or implied.
% See the License for the specific language governing permissions and
% limitations under the License.

\hypertarget{tool-features}{%
\section{Tool features}\label{tool-features}}

The \pkg\ facilitates seamless integration and synchronization between multiple 
issue tracking platforms. The main operations include:

\begin{enumerate}
   \item Configuration Parsing
         \begin{itemize}
            \item Reads the JSON configuration file to understand the 
                  synchronization scope and behavior.
         \end{itemize}

   \item Issue Collection
         \begin{itemize}
            \item Fetches issues from GitHub, Gitlab and Jira based on the 
                  specified conditions.
            \item Uses user mappings to ensure issues are associated correctly 
                  across platforms.
         \end{itemize}

   \item Issue Update
         \begin{itemize}
            \item Updates the source issues with RTC IDs and planning data after 
                  synchronization.
         \end{itemize}

   \item Synchronization to RTC
         \begin{itemize}
            \item Creates or updates work items in RTC with the collected issues.
            \item Includes planning data provided by RTC.
         \end{itemize}
\end{enumerate}

\hypertarget{tool-usage}{%
\section{Tool usage}\label{tool-usage}}

Use below command to get tools's usage:

\begin{pythonlog}
IssueSyncTool -h
\end{pythonlog}

The tool's usage should be showed as below:

\begin{pythonlog}
usage: IssueSyncTool (Tickets Sync Tool) [-h] --config CONFIG [--dryrun] [--csv] [-v]

IssueSyncTool sync ticket|issue|workitem between tracking systems such as Github Issue, JIRA and IBM RTC

optional arguments:
  -h, --help       show this help message and exit
  --config CONFIG  path to configuration json file
  --dryrun         if set, then just dump the tickets without syncing
  --csv            if set, then store the sync status to csv file sync_status.csv
  -v, --version    version of the IssueSyncTool
\end{pythonlog}

Sample command to run \pkg\ with the configuration JSON file and save sync 
status as csv file:

\begin{pythonlog}
IssueSyncTool --config <your-config-file> --csv
\end{pythonlog}

\hypertarget{config-file}{%
\section{JSON Configuration File}\label{config-file}}

The tool uses a JSON configuration file to define synchronization behavior. 
Below is an explanation of the sample configuration:

\subsection{Source and Destination Platforms}
\begin{pythoncode}
{
   "source": ["github", "gitlab", jira"],
   "destination": ["rtc"]
   ...
}
\end{pythoncode}
This configuration specifies GitHub and Jira as sources and RTC as the 
destination for synchronization.

\subsection{Tracker Configurations}
\begin{pythoncode}
{
   ...
   "tracker": {
      "github": {
         "project" : "test-fullautomation",
         "token": "<your-PAT>",
         "repository": [
            "python-jsonpreprocessor"
         ],
         "condition": {
            "state": "open",
            "exclude": {
               "assignee": "empty",
               "labels": "0.13.1"
            }
         }
      },
      ...
   }
   ... 
}
\end{pythoncode}

Above code is sample configuration of Github tracker which contain the 
information about:

\begin{itemize}
    \item \textbf{Project:} Github project \pcode{"test-fullautomation"}
    \item \textbf{Token:} A personal access token for authentication.
    \item \textbf{Repositories:} List of repositories to collect issues from.
    \item \textbf{Condition:} Define the condition (query) to collect the issues.
          \begin{itemize}
            \item \pcode{"state"}: Syncs only issues in the specified state, 
                  e.g., \pcode{"open"}.
            \item \pcode{"exclude"}: Specifies negative conditions. For example:
            \begin{itemize}
               \item \pcode{"assignee": "empty"}: Excludes issues with no assignee.
               \item \pcode{"labels": "0.13.1"}: Excludes issues labeled \pcode{"0.13.1"}.
            \end{itemize}
          \end{itemize}
\end{itemize}

The other tracker can be configured as the same way.

\subsection{User Mapping}
User mapping ensures that the correct user is assigned in the synchronization 
process across different platforms. 
In the configuration file, each user is mapped to their corresponding accounts 
across GitHub, Jira, Gitlab and RTC. This mapping helps to ensure that the right 
assignee is applied to issues in the appropriate tracker system.

\begin{itemize}
   \item The \texttt{user} section of the configuration file specifies the 
         mapping between the users' names in GitHub, Gitlab, Jira and RTC. 
   \item This ensures that the correct user is set as the assignee in each 
         platform when syncing issues.
   \item For instance, when syncing an issue from Jira to RTC, the tool will 
         automatically assign the same user (as per the mapping) to the issue in RTC. 
   \item If the user has different usernames across platforms 
         (e.g., "githubUser" in GitHub, "jiraUser" in Jira, and "rtcUser" in RTC), 
         the tool ensures the correct mapping is applied so that all systems 
         reflect the same assignee.
\end{itemize}

Example configuration:
\begin{pythoncode}
{
   ... 
   "user": [
      {
         "name": "Tran Duy Ngoan",
         "github": "ngoan1608",
         "jira": "ntd1hc",
         "gitlab": "ntd1hc",
         "rtc": "ntd1hc"
      },
      ... 
   ]   
}
\end{pythoncode}

In this example:
\begin{itemize}
    \item The user \texttt{Tran Duy Ngoan} is mapped to \texttt{ngoan1608} in 
          GitHub, \texttt{ntd1hc} in Jira, and \texttt{ntd1hc} in RTC.
    \item When syncing issues between GitHub, Jira, and RTC, the tool ensures 
          that issues assigned to \texttt{ngoan1608} in GitHub and 
          \texttt{ntd1hc} in Jira will be assigned to \texttt{ntd1hc} in RTC, 
          ensuring consistent user data across all platforms.
\end{itemize}